\documentclass[mgr, pl, oneside, openright, final, openbib]{mgr}
%\documentclass[eng, pl, oneside, openright, final, openbib]
\DeclareUnicodeCharacter{0301}{\'{e}}
\usepackage{polski}
\usepackage[utf8]{inputenc}
\usepackage{comment}
\usepackage{babel}
\usepackage{graphics}
\usepackage{lscape}
\usepackage{url}
\usepackage{amsmath}
\usepackage{plain}
\usepackage{amsfonts}
\usepackage{multirow}
\usepackage{listings}
\newtheorem{twr}{Twierdzenie}
\def\listtablename{Spis tabel}
\def\tablename{Tabela. }
\usepackage{tikz}
\usepackage{array}
\usepackage{algorithmicx}
\usepackage{makeidx}
\usetikzlibrary{shapes.geometric, arrows}
\author{Kowalczyk Konrad}
\title{Analiza porównawcza zaimplementowanych algorytmów rozwiązujących problem plecakowy z rozwiązaniami zaproponowanych w wybranych artykułach naukowych}
\engtitle{Comparative analysis of the implemented algorithms solving the knapsack problem with the solutions proposed in scientific articles}
\supervisor{Dr inż. Marcin Jaroszewski W4n}
\field{Teleinformatyka (TIN)}
\specialisation{Utrzymanie sieci teleinformatycznych (TIU)}
\date{2023}

\begin{document}
\maketitle
\pagenumbering{arabic}
\tableofcontents

\chapter{Wprowadzenie}

\section{Cel pracy dyplomowej}
\section{Założenia oraz zakres pracy dyplomowej}
\section{Wykorzystywane narzędzia}

\newpage
\chapter{Metody rozwiązywania problemów optymalizacyjnych}
\section{Problem plecakowy - Knapsack Problem}
\section{Rozwiązania dokładne}
\section{Rozwiązania przybliżone}
\newpage
\chapter{Proponowane rozwiązania problemu pracy dyplomowej}
\newpage
\chapter{Rozwiązania zaproponowane przez autora tekstu}
\newpage
\chapter{Badania otrzymanych rozwiązań}
\newpage
\chapter{Porównanie wyników pomiędzy różnymi implementacjami}
\newpage
\chapter{Wnioski}
\newpage
\chapter{Podsumowanie}
\listoffigures

\end{document}